\documentclass[10pt,fullpage, a4paper, titlepage]{article}
\usepackage{graphicx, latexsym}
\graphicspath{{images/}} %configuring the graphicx package
\usepackage{setspace} 
\usepackage{apalike}
\usepackage{amssymb, amsmath, amsthm}
\usepackage{bm}
\usepackage{epstopdf}
\usepackage{subcaption}  % For subfigures
\usepackage{caption}     % For captions
\usepackage{float}       % For [H] function
\usepackage[]{hyperref}
\hypersetup{
    pdftitle={LaTeX week 1},
    pdfauthor={Sophie Hetche},
    pdfsubject={exercise week 1},
    pdfkeywords={equation, table, figure},
    bookmarksnumbered=true,     
    bookmarksopen=true,         
    bookmarksopenlevel=1,       
    colorlinks=true, 
    linkcolor=blue,  % Color of internal links
    pdfstartview=Fit,           
    pdfpagemode=UseOutlines,      
    pdfpagelayout=TwoPageRight
}

\singlespacing
%\onehalfspacing
%\doublespacing

\title{\textbf{Week 1:} Introduction to $\text{\LaTeX}$\\ \small The first exercise}
\author{Sophie Hetche (9058915)}
\date{\today}
%\date{}


\begin{document}
\maketitle
\tableofcontents % This command creates the Table of Contents
\newpage

\section{The Equation}

The solutions of a quadratic equation can be expressed using the \textbf{p-q formula:} \\

The p-q formula is described by the equation  $x = \frac{-p}{2} \pm \sqrt{\left(\frac{p}{2}\right)^2 - q}$. Here, $p$ is always the number in front of $x$, and $q$ is the constant term without $x$.\\

I can also present the equation in \textbf{display math mode}:
\[
x = \frac{-p}{2} \pm \sqrt{\left(\frac{p}{2}\right)^2 - q}
\]
I will never forget this formula simply because my math teacher played \href{https://www.youtube.com/watch?v=cmUHBZWZfrs}{this German kids rap song}  to us in 7th grade, and the melody stuck so I will always now how to solve a quadratic equation.


\section{Figures and Tables}

\subsection{Here we see four different plots}
\begin{figure}[htbp]
    \centering
    % First row of images
    \begin{subfigure}[b]{0.45\textwidth} %figure 45% of the text width
        \centering %centering subfigures within big figure environment
        %scale image to the width of the subfigure:
        \includegraphics[width=\linewidth]{images/histogram.png} 
        \caption{Histogram} %subfigure package automat. labels (a), etc.
        \label{fig:histogram}
    \end{subfigure}
    \hfill %horizontal space between subfigures on the same line
    \begin{subfigure}[b]{0.45\textwidth}
        \centering 
        \includegraphics[width=\linewidth]{images/densityplot.png}
        \caption{Density Plot}
        \label{fig:density}
    \end{subfigure}
    
    \vspace{1em} % vertical space between the two rows of subfigures

    % Second row of images
    \begin{subfigure}[b]{0.45\textwidth}
        \centering 
        \includegraphics[width=\linewidth]{images/stripplot.png}
        \caption{Strip Plot}
        \label{fig:stripplot}
    \end{subfigure}
    \hfill 
    \begin{subfigure}[b]{0.45\textwidth}
        \centering 
        \includegraphics[width=\linewidth]{images/bwplot.png}
        \caption{Box Plot}
        \label{fig:boxplot}
    \end{subfigure}
    %overall caption for the entire figure
    \caption{Different plots of the same data, sometimes transformed.}
    \label{fig:overview}
\end{figure}

\subsection{Here we see a table}
% latex table generated in R 4.4.1 by xtable 1.8-4 package
% Mon Oct  7 17:58:42 2024
\begin{table}[H] %force table to stay in the position where it is defined.
\centering
\caption{The same data displayed in a table.}
\label{tab:data_summary}
\begin{tabular}{rrrrr}
  \hline
 & data & squared1 & squared2 & exponent \\ 
  \hline
1 & -0.56 & 0.31 & 0.31 & 0.57 \\ 
  2 & -0.23 & 0.05 & 0.05 & 0.79 \\ 
  3 & 1.56 & 2.43 & 2.43 & 4.75 \\ 
  4 & 0.07 & 0.00 & 0.00 & 1.07 \\ 
  5 & 0.13 & 0.02 & 0.02 & 1.14 \\ 
  6 & 1.72 & 2.94 & 2.94 & 5.56 \\ 
  7 & 0.46 & 0.21 & 0.21 & 1.59 \\ 
  8 & -1.27 & 1.60 & 1.60 & 0.28 \\ 
  9 & -0.69 & 0.47 & 0.47 & 0.50 \\ 
   \hline
\end{tabular}
\begin{center}
\textit{Note: Only the first nine rows are displayed.}
\end{center}
\end{table}


\section{The Fairy Tale: The Enchanted Garden and the Magical Dove}

Once, in a land covered by mists and whispers, there lay an enchanting garden hidden behind a great stone wall. No one knew who had built the wall or why, but one thing was for certain – nobody had ever seen what was behind it.

A little boy named Peter lived in a village nearby. Fuelled by curiosity and tales of magical creatures, he often dreamt of the wonders that the walled garden might hold. One day, unable to resist its lure any longer, he decided to find a way in.

As he approached the towering stone barrier, he noticed a tiny gap just big enough for him to peek through. The garden inside was bathed in a shimmering golden light, unlike any he had ever seen. To his amazement, in the center stood a magnificent tree with leaves that glittered as if they were made of starlight. And resting on one of its branches was a dove, glowing with the same luminous hue.

Before he could process this beautiful sight, the dove spoke to his in a voice as soft as the wind, "To enter the garden, one must share a pure and selfless desire."

Peter, with his heart pounding, whispered his wish, "I wish for everyone in my village to be happy and free from suffering."

The massive stone door, seemingly of its own accord, began to open. The luminous dove flew to Peter and rested on his shoulder. "Your wish is genuine, and so you may enter," it said.

Inside, the garden was more wondrous than Peter had ever imagined. Flowers sang in soft harmonies, and a gentle breeze carried the sweetest of fragrances. Every step he took made the grass shimmer with colors she'd never seen before.

The dove explained that this was an Enchanted Garden, a place where one’s purest wishes could come true. But, there was a catch. To make his wish a reality, Peter had to plant a seed from the magical tree in his village and care for it with unwavering love and dedication.

Peter accepted the challenge. With the seed safely tucked in his pocket and the dove guiding her, he returned to his village.

Years went by, and with Peter's love, the seed grew into a magnificent tree, similar to the one in the Enchanted Garden. With its growth, joy and happiness blossomed in the village like never before.

And so, in a village once shadowed by mystery, there stood a tree that bore witness to the pure heart of a boy and his luminous companion, reminding everyone that magic was always just a wish away.

\end{document}
