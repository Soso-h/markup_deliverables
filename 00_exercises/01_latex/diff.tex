\documentclass[10pt,fullpage, a4paper, titlepage]{article}
%DIF LATEXDIFF DIFFERENCE FILE
%DIF DEL original_document.tex   Tue Oct  8 17:05:22 2024
%DIF ADD modified_document.tex   Tue Oct  8 17:05:17 2024
\usepackage{graphicx, latexsym}
\graphicspath{{images/}} %configuring the graphicx package
\usepackage{setspace} 
\usepackage{apalike}
\usepackage{amssymb, amsmath, amsthm}
\usepackage{bm}
\usepackage{epstopdf}
\usepackage{subcaption}  % For subfigures
\usepackage{caption}     % For captions
\usepackage{float}       % For [H] function
\usepackage[]{hyperref}
\hypersetup{
    pdftitle={LaTeX week 1},
    pdfauthor={Sophie Hetche},
    pdfsubject={exercise week 1},
    pdfkeywords={equation, table, figure},
    bookmarksnumbered=true,     
    bookmarksopen=true,         
    bookmarksopenlevel=1,       
    colorlinks=true, 
    linkcolor=blue,  % Color of internal links
    pdfstartview=Fit,           
    pdfpagemode=UseOutlines,      
    pdfpagelayout=TwoPageRight
}

\singlespacing
%\onehalfspacing
%\doublespacing

\title{\textbf{Week 1:} Introduction to $\text{\LaTeX}$\\ \small The first exercise}
\author{Sophie Hetche (9058915)}
\date{\today}
%\date{}
%DIF PREAMBLE EXTENSION ADDED BY LATEXDIFF
%DIF UNDERLINE PREAMBLE %DIF PREAMBLE
\RequirePackage[normalem]{ulem} %DIF PREAMBLE
\RequirePackage{color}\definecolor{RED}{rgb}{1,0,0}\definecolor{BLUE}{rgb}{0,0,1} %DIF PREAMBLE
\providecommand{\DIFaddtex}[1]{{\protect\color{blue}\uwave{#1}}} %DIF PREAMBLE
\providecommand{\DIFdeltex}[1]{{\protect\color{red}\sout{#1}}}                      %DIF PREAMBLE
%DIF SAFE PREAMBLE %DIF PREAMBLE
\providecommand{\DIFaddbegin}{} %DIF PREAMBLE
\providecommand{\DIFaddend}{} %DIF PREAMBLE
\providecommand{\DIFdelbegin}{} %DIF PREAMBLE
\providecommand{\DIFdelend}{} %DIF PREAMBLE
\providecommand{\DIFmodbegin}{} %DIF PREAMBLE
\providecommand{\DIFmodend}{} %DIF PREAMBLE
%DIF FLOATSAFE PREAMBLE %DIF PREAMBLE
\providecommand{\DIFaddFL}[1]{\DIFadd{#1}} %DIF PREAMBLE
\providecommand{\DIFdelFL}[1]{\DIFdel{#1}} %DIF PREAMBLE
\providecommand{\DIFaddbeginFL}{} %DIF PREAMBLE
\providecommand{\DIFaddendFL}{} %DIF PREAMBLE
\providecommand{\DIFdelbeginFL}{} %DIF PREAMBLE
\providecommand{\DIFdelendFL}{} %DIF PREAMBLE
%DIF HYPERREF PREAMBLE %DIF PREAMBLE
\providecommand{\DIFadd}[1]{\texorpdfstring{\DIFaddtex{#1}}{#1}} %DIF PREAMBLE
\providecommand{\DIFdel}[1]{\texorpdfstring{\DIFdeltex{#1}}{}} %DIF PREAMBLE
\newcommand{\DIFscaledelfig}{0.5}
%DIF HIGHLIGHTGRAPHICS PREAMBLE %DIF PREAMBLE
\RequirePackage{settobox} %DIF PREAMBLE
\RequirePackage{letltxmacro} %DIF PREAMBLE
\newsavebox{\DIFdelgraphicsbox} %DIF PREAMBLE
\newlength{\DIFdelgraphicswidth} %DIF PREAMBLE
\newlength{\DIFdelgraphicsheight} %DIF PREAMBLE
% store original definition of \includegraphics %DIF PREAMBLE
\LetLtxMacro{\DIFOincludegraphics}{\includegraphics} %DIF PREAMBLE
\newcommand{\DIFaddincludegraphics}[2][]{{\color{blue}\fbox{\DIFOincludegraphics[#1]{#2}}}} %DIF PREAMBLE
\newcommand{\DIFdelincludegraphics}[2][]{% %DIF PREAMBLE
\sbox{\DIFdelgraphicsbox}{\DIFOincludegraphics[#1]{#2}}% %DIF PREAMBLE
\settoboxwidth{\DIFdelgraphicswidth}{\DIFdelgraphicsbox} %DIF PREAMBLE
\settoboxtotalheight{\DIFdelgraphicsheight}{\DIFdelgraphicsbox} %DIF PREAMBLE
\scalebox{\DIFscaledelfig}{% %DIF PREAMBLE
\parbox[b]{\DIFdelgraphicswidth}{\usebox{\DIFdelgraphicsbox}\\[-\baselineskip] \rule{\DIFdelgraphicswidth}{0em}}\llap{\resizebox{\DIFdelgraphicswidth}{\DIFdelgraphicsheight}{% %DIF PREAMBLE
\setlength{\unitlength}{\DIFdelgraphicswidth}% %DIF PREAMBLE
\begin{picture}(1,1)% %DIF PREAMBLE
\thicklines\linethickness{2pt} %DIF PREAMBLE
{\color[rgb]{1,0,0}\put(0,0){\framebox(1,1){}}}% %DIF PREAMBLE
{\color[rgb]{1,0,0}\put(0,0){\line( 1,1){1}}}% %DIF PREAMBLE
{\color[rgb]{1,0,0}\put(0,1){\line(1,-1){1}}}% %DIF PREAMBLE
\end{picture}% %DIF PREAMBLE
}\hspace*{3pt}}} %DIF PREAMBLE
} %DIF PREAMBLE
\LetLtxMacro{\DIFOaddbegin}{\DIFaddbegin} %DIF PREAMBLE
\LetLtxMacro{\DIFOaddend}{\DIFaddend} %DIF PREAMBLE
\LetLtxMacro{\DIFOdelbegin}{\DIFdelbegin} %DIF PREAMBLE
\LetLtxMacro{\DIFOdelend}{\DIFdelend} %DIF PREAMBLE
\DeclareRobustCommand{\DIFaddbegin}{\DIFOaddbegin \let\includegraphics\DIFaddincludegraphics} %DIF PREAMBLE
\DeclareRobustCommand{\DIFaddend}{\DIFOaddend \let\includegraphics\DIFOincludegraphics} %DIF PREAMBLE
\DeclareRobustCommand{\DIFdelbegin}{\DIFOdelbegin \let\includegraphics\DIFdelincludegraphics} %DIF PREAMBLE
\DeclareRobustCommand{\DIFdelend}{\DIFOaddend \let\includegraphics\DIFOincludegraphics} %DIF PREAMBLE
\LetLtxMacro{\DIFOaddbeginFL}{\DIFaddbeginFL} %DIF PREAMBLE
\LetLtxMacro{\DIFOaddendFL}{\DIFaddendFL} %DIF PREAMBLE
\LetLtxMacro{\DIFOdelbeginFL}{\DIFdelbeginFL} %DIF PREAMBLE
\LetLtxMacro{\DIFOdelendFL}{\DIFdelendFL} %DIF PREAMBLE
\DeclareRobustCommand{\DIFaddbeginFL}{\DIFOaddbeginFL \let\includegraphics\DIFaddincludegraphics} %DIF PREAMBLE
\DeclareRobustCommand{\DIFaddendFL}{\DIFOaddendFL \let\includegraphics\DIFOincludegraphics} %DIF PREAMBLE
\DeclareRobustCommand{\DIFdelbeginFL}{\DIFOdelbeginFL \let\includegraphics\DIFdelincludegraphics} %DIF PREAMBLE
\DeclareRobustCommand{\DIFdelendFL}{\DIFOaddendFL \let\includegraphics\DIFOincludegraphics} %DIF PREAMBLE
%DIF COLORLISTINGS PREAMBLE %DIF PREAMBLE
\RequirePackage{listings} %DIF PREAMBLE
\RequirePackage{color} %DIF PREAMBLE
\lstdefinelanguage{DIFcode}{ %DIF PREAMBLE
%DIF DIFCODE_UNDERLINE %DIF PREAMBLE
  moredelim=[il][\color{red}\sout]{\%DIF\ <\ }, %DIF PREAMBLE
  moredelim=[il][\color{blue}\uwave]{\%DIF\ >\ } %DIF PREAMBLE
} %DIF PREAMBLE
\lstdefinestyle{DIFverbatimstyle}{ %DIF PREAMBLE
	language=DIFcode, %DIF PREAMBLE
	basicstyle=\ttfamily, %DIF PREAMBLE
	columns=fullflexible, %DIF PREAMBLE
	keepspaces=true %DIF PREAMBLE
} %DIF PREAMBLE
\lstnewenvironment{DIFverbatim}{\lstset{style=DIFverbatimstyle}}{} %DIF PREAMBLE
\lstnewenvironment{DIFverbatim*}{\lstset{style=DIFverbatimstyle,showspaces=true}}{} %DIF PREAMBLE
%DIF END PREAMBLE EXTENSION ADDED BY LATEXDIFF

\begin{document}
\maketitle
\tableofcontents % This command creates the Table of Contents
\newpage

\section{The Equation}

The solutions of a quadratic equation can be expressed using the \textbf{p-q formula:} \\

The p-q formula is described by the equation  $x = \frac{-p}{2} \pm \sqrt{\left(\frac{p}{2}\right)^2 - q}$. Here, $p$ is always the number in front of $x$, and $q$ is the constant term without $x$.\\

I can also present the equation in \textbf{display math mode}:
\[
x = \frac{-p}{2} \pm \sqrt{\left(\frac{p}{2}\right)^2 - q}
\]
I will never forget this formula simply because my math teacher played \href{https://www.youtube.com/watch?v=cmUHBZWZfrs}{this German kids rap song}  to us in 7th grade, and the melody stuck so I will always now how to solve a quadratic equation.


\section{Figures and Tables}

\subsection{Here we see four different plots}
\begin{figure}[htbp]
    \centering
    % First row of images
    \begin{subfigure}[b]{0.45\textwidth} %figure 45% of the text width
        \centering %centering subfigures within big figure environment
        %scale image to the width of the subfigure:
        \includegraphics[width=\linewidth]{images/histogram.png} 
        \caption{Histogram} %subfigure package automat. labels (a), etc.
        \label{fig:histogram}
    \end{subfigure}
    \hfill %horizontal space between subfigures on the same line
    \begin{subfigure}[b]{0.45\textwidth}
        \centering 
        \includegraphics[width=\linewidth]{images/densityplot.png}
        \caption{Density Plot}
        \label{fig:density}
    \end{subfigure}

    \vspace{1em} % vertical space between the two rows of subfigures

    % Second row of images
    \begin{subfigure}[b]{0.45\textwidth}
        \centering 
        \includegraphics[width=\linewidth]{images/stripplot.png}
        \caption{Strip Plot}
        \label{fig:stripplot}
    \end{subfigure}
    \hfill 
    \begin{subfigure}[b]{0.45\textwidth}
        \centering 
        \includegraphics[width=\linewidth]{images/bwplot.png}
        \caption{Box Plot}
        \label{fig:boxplot}
    \end{subfigure}
    %overall caption for the entire figure
    \caption{Different plots of the same data, sometimes transformed.}
    \label{fig:overview}
\end{figure}

\subsection{Here we see a table}
% latex table generated in R 4.4.1 by xtable 1.8-4 package
% Mon Oct  7 17:58:42 2024
\begin{table}[H] %force table to stay in the position where it is defined.
\centering
\caption{The same data displayed in a table.}
\label{tab:data_summary}
\begin{tabular}{rrrrr}
  \hline
 & data & squared1 & squared2 & exponent \\ 
  \hline
1 & -0.56 & 0.31 & 0.31 & 0.57 \\ 
  2 & -0.23 & 0.05 & 0.05 & 0.79 \\ 
  3 & 1.56 & 2.43 & 2.43 & 4.75 \\ 
  4 & 0.07 & 0.00 & 0.00 & 1.07 \\ 
  5 & 0.13 & 0.02 & 0.02 & 1.14 \\ 
  6 & 1.72 & 2.94 & 2.94 & 5.56 \\ 
  7 & 0.46 & 0.21 & 0.21 & 1.59 \\ 
  8 & -1.27 & 1.60 & 1.60 & 0.28 \\ 
  9 & -0.69 & 0.47 & 0.47 & 0.50 \\ 
   \hline
\end{tabular}
\begin{center}
\textit{Note: Only the first nine rows are displayed.}
\end{center}
\end{table}


\section{The Fairy Tale: The Enchanted Garden and the \DIFdelbegin \DIFdel{Luminous }\DIFdelend \DIFaddbegin \DIFadd{Magical }\DIFaddend Dove}

Once, in a land covered by mists and whispers, there lay an enchanting garden hidden behind a great stone wall. No one knew who had built the wall or why, but one thing was for certain – nobody had ever seen what was behind it.

A little \DIFdelbegin \DIFdel{girl named Clara }\DIFdelend \DIFaddbegin \DIFadd{boy named Peter }\DIFaddend lived in a village nearby. \DIFdelbegin \DIFdel{Fueled }\DIFdelend \DIFaddbegin \DIFadd{Fuelled }\DIFaddend by curiosity and tales of magical creatures, \DIFdelbegin \DIFdel{she }\DIFdelend \DIFaddbegin \DIFadd{he }\DIFaddend often dreamt of the wonders that the walled garden might hold. One day, unable to resist its lure any longer, \DIFdelbegin \DIFdel{she }\DIFdelend \DIFaddbegin \DIFadd{he }\DIFaddend decided to find a way in.

As \DIFdelbegin \DIFdel{she }\DIFdelend \DIFaddbegin \DIFadd{he }\DIFaddend approached the towering stone barrier, \DIFdelbegin \DIFdel{she }\DIFdelend \DIFaddbegin \DIFadd{he }\DIFaddend noticed a tiny gap just big enough for \DIFdelbegin \DIFdel{her }\DIFdelend \DIFaddbegin \DIFadd{him }\DIFaddend to peek through. The garden inside was bathed in a shimmering golden light, unlike any \DIFdelbegin \DIFdel{she }\DIFdelend \DIFaddbegin \DIFadd{he }\DIFaddend had ever seen. To \DIFdelbegin \DIFdel{her }\DIFdelend \DIFaddbegin \DIFadd{his }\DIFaddend amazement, in the center stood a magnificent tree with leaves that glittered as if they were made of starlight. And resting on one of its branches was a dove, glowing with the same luminous hue.

Before \DIFdelbegin \DIFdel{she }\DIFdelend \DIFaddbegin \DIFadd{he }\DIFaddend could process this beautiful sight, the dove spoke to \DIFdelbegin \DIFdel{her }\DIFdelend \DIFaddbegin \DIFadd{his }\DIFaddend in a voice as soft as the wind, "To enter the garden, one must share a pure and selfless desire."

\DIFdelbegin \DIFdel{Clara, with her }\DIFdelend \DIFaddbegin \DIFadd{Peter, with his }\DIFaddend heart pounding, whispered \DIFdelbegin \DIFdel{her }\DIFdelend \DIFaddbegin \DIFadd{his }\DIFaddend wish, "I wish for everyone in my village to be happy and free from suffering."

The massive stone door, seemingly of its own accord, began to open. The luminous dove flew to \DIFdelbegin \DIFdel{Clara }\DIFdelend \DIFaddbegin \DIFadd{Peter }\DIFaddend and rested on \DIFdelbegin \DIFdel{her }\DIFdelend \DIFaddbegin \DIFadd{his }\DIFaddend shoulder. "Your wish is genuine, and so you may enter," it said.

Inside, the garden was more wondrous than \DIFdelbegin \DIFdel{Clara }\DIFdelend \DIFaddbegin \DIFadd{Peter }\DIFaddend had ever imagined. Flowers sang in soft harmonies, and a gentle breeze carried the sweetest of fragrances. Every step \DIFdelbegin \DIFdel{she }\DIFdelend \DIFaddbegin \DIFadd{he }\DIFaddend took made the grass shimmer with colors she'd never seen before.

The dove explained that this was an Enchanted Garden, a place where one’s purest wishes could come true. But, there was a catch. To make \DIFdelbegin \DIFdel{her }\DIFdelend \DIFaddbegin \DIFadd{his }\DIFaddend wish a reality, \DIFdelbegin \DIFdel{Clara }\DIFdelend \DIFaddbegin \DIFadd{Peter }\DIFaddend had to plant a seed from the magical tree in \DIFdelbegin \DIFdel{her }\DIFdelend \DIFaddbegin \DIFadd{his }\DIFaddend village and care for it with unwavering love and dedication.

\DIFdelbegin \DIFdel{Clara }\DIFdelend \DIFaddbegin \DIFadd{Peter }\DIFaddend accepted the challenge. With the seed safely tucked in \DIFdelbegin \DIFdel{her }\DIFdelend \DIFaddbegin \DIFadd{his }\DIFaddend pocket and the dove guiding her, \DIFdelbegin \DIFdel{she returned to her }\DIFdelend \DIFaddbegin \DIFadd{he returned to his }\DIFaddend village.

Years went by, and with \DIFdelbegin \DIFdel{Clara}\DIFdelend \DIFaddbegin \DIFadd{Peter}\DIFaddend 's love, the seed grew into a magnificent tree, similar to the one in the Enchanted Garden. With its growth, joy and happiness blossomed in the village like never before.

\DIFdelbegin \DIFdel{Clara's selfless wish not only transformed her village but also changed her. She became known as the Keeper of Joy, teaching future generations about love, compassion, and the magic of selfless wishes.
}%DIFDELCMD < 

%DIFDELCMD < %%%
\DIFdelend And so, in a village once shadowed by mystery, there stood a tree that bore witness to the pure heart of a \DIFdelbegin \DIFdel{girl and her }\DIFdelend \DIFaddbegin \DIFadd{boy and his }\DIFaddend luminous companion, reminding everyone that magic was always just a wish away.

\end{document}
